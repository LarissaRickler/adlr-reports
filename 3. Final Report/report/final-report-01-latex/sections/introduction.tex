Vertical take-off and landing (VTOL) drone are extremely versatile. Possible application fields are logistics, infrastructure, search, rescue and many more \cite{Drone_Applications}. Models, such as quadrotors, are highly agile and allow for rapid maneuvers \cite{Maneuverability_Agility_Art, Maneuverability_Agility_Exp}. Hence, they are ideal for survivor search in mountains or after natural disasters like earthquakes, floods or wildfires. There is also a huge potential in autonomous flights \cite{Autonomic_Flights}. 

%In order to exploit the full potential of drones, it is necessary to explore the boundaries of their performance in maneuvering without a human operator. Hence, racing competitions that aim to minimize the flight time though a sequence of waypoints are of great relevance for further research developments in this field. By the nature of racing, the drones are forced to operate close to their aerodynamic boundaries and are pushed towards their performance limits. This provides high requirements for the quality and robustness of the proposed controller.

Deep reinforcement learning methods have been successfully utilized to train a neural network controllers for minimum-time quadrotor flight \cite{Penicka_2022}. This research adapted the results from Penicka et al. to an low-level controller that directly controls each rotor individually without the need of a underlying PID controller.